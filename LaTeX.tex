\documentclass[a4paper,12pt]{article}
\usepackage{fullpage}
\usepackage{graphicx}
\usepackage{listings}
\usepackage{xcolor}

%This allows us to insert LaTeX code snippets
\lstset{
   	language=[LaTeX]TeX,
	aboveskip=5mm,
	belowskip=5mm,
 	breaklines=true,
	%basicstyle=\large\scriptsize,
   	keywordstyle=\color{blue},
   	identifierstyle=\color{magenta},
}

\begin{document}

%Title Page
\begin{titlepage}
	\begin{figure}[!ht]
		\centering
			\includegraphics[scale = 0.1]{logo}
	\end{figure}
	\vspace{3cm}
	\centering
	{\huge By Team \par}
	\vspace{0.5cm}
	{\huge\itshape Can I Eat LaTeX \par}
	\vspace{2cm}
	{\scshape\Large  Members: Jason Shin, Johnny So, Kevin Yan\par}
	\vspace{\stretch{1}}
	{\large Software Development Pd. 6}
\end{titlepage}
	
%Johnny's Part

\newpage
\begin{center}
  	\bf{\Huge{"Short" Introduction}}
\end{center}
\bigskip
\Large
\textbf{\TeX} (pronouned tech), created by Donald E. Knuth, is a program that is
designed to typeset text and mathematical formulae. \\\\
\textbf{\LaTeX} (pronounced luh-tech) is a kind of dialect of TeX, and is aimed to help beginners/authors to
format their work in a professional manner. \\\\
Together, the writer plays the role of an author, \LaTeX\ plays the role of a book designer, and TeX is the
typesetter. It is quite different from WYSIWYG (what you see is what you get)
word processors like \textbf{Microsoft Word}. You generally cannot see, in real
time, what your document will look like as you work on it in \LaTeX\ since you
have to compile it. \\\\
There are numerous ways to do so - programs like pdflatex and MiKTeX will
compile your \LaTeX\ file into a PDF, and there are packages/extensions in Emacs, Atom, etc.
that compile it from within the environment.

\newpage
\begin{center}
	\bf{\Huge{"Short" Introduction (cont.)}}
\end{center}
In addition, LaTeX is considered to be a programming language. \\
Various aspects of LaTeX are very programm-y. \\\\
For example, you probably already noticed
how \verb|\users| have a section at the top of their base file. \\\\
The \verb|"\usepackage"| command essentially functions as an import from an existing flo \
In addition, users can write thier own ``commands''.

%Kevin's Part

%Intro/Summary Page
\newpage
\begin{center}
	\bf{\Huge{The Cool Stuff}}
\end{center}
\bigskip
\LARGE
Now that we've gone over the basics, let's take a look at some of the other cool things you can do with LaTeX!
\begin{itemize}
	\item Graphics
	\item Packages
	\item Custom commands and environments
\end{itemize}
\bigskip
\Large
LaTeX has the capability to do almost anything that you'd need from a word processor. \\\\
In order to implement most of the cool features available in \\ LaTeX, we'll need to use packages. \\\\
To use a package, similar to imports in Java or Python, we simply declare \verb|\usepackage{packagename}| at the top of our file. \\\\
Afterwards, we can use all the functions that package provides. \\\\
There are a few essential packages that most people will end up using. \\\\
For example, the package \textit{fullpage} will set the margins of the paper to normal 1x1 instead of math paper margins.

%Graphicx Page
\newpage
\begin{center}
  	\bf{\Huge{Pictures}}
\end{center}
\bigskip
\Large
The easiest way to insert a picture in LaTeX is with the package \textit{graphicx}. \\
\begin{figure}[h]
	\centering
		\includegraphics[scale = 0.5]{heavy_breathing}
\end{figure}
\large
\begin{verbatim}
	\usepackage{graphicx}
	.....
	\begin{figure}[h]
		\centering
		\includegraphics[scale = 0.5]{heavy_breathing}
	\end{figure}
\end{verbatim}
\bigskip
Here, the \verb|[h]| is an additional argument that can be given to the environment. \\\\
h means to display the image \textit{here} on the page. Images enclosed in figure environments will float to the top of the page by default. \\\\
You can also use the arguments $t$ for top, $b$ for bottom, etc. \\\\
And you can use more than one argument in order of preference in case one of them fails: \verb|[htb]|

%Code Snippets Page
\newpage
\begin{center}
	\bf{\Huge{lstlistings}}
\end{center}
\bigskip
Here's a code snippet which contains the code needed to insert a code snippet into LaTeX.
\begin{lstlisting}
	\usepackage{listings}
	\usepackage{xcolor}
	.....
	\lstset{
		language=[LaTeX]TeX,
		breaklines=true,
		basicstyle=\tt\scriptsize,
		keywordstyle=\color{blue},
		identifierstyle=\color{magenta},
	}
	.....
	\begin{lstlisting}
		.....
\end{lstlisting}
\end{document}
