\documentclass[a4paper,12pt]{article}
\usepackage{fullpage}
\usepackage{graphicx}
\usepackage{listings}
\usepackage{xcolor}
\usepackage{siunitx}

%This allows us to insert LaTeX code snippets
\lstset{
   	language=[LaTeX]TeX,
	aboveskip=5mm,
	belowskip=5mm,
 	breaklines=true,
	%basicstyle=\large\scriptsize,
   	keywordstyle=\color{blue},
   	identifierstyle=\color{magenta},
}

\begin{document}

%Title Page
\begin{titlepage}
	\begin{figure}[!ht]
		\centering
			\includegraphics[scale = 0.1]{logo}
	\end{figure}
	\vspace{3cm}
	\centering
	{\huge By Team \par}
	\vspace{0.5cm}
	{\huge\itshape Can I Eat LaTeX \par}
	\vspace{2cm}
	{\scshape\Large  Members: Jason Shin, Johnny So, Kevin Yan\par}
	\vspace{\stretch{1}}
	{\large Software Development Pd. 6}
\end{titlepage}
	
%Johnny's Part

\newpage
\begin{center}
  	\bf{\Huge{"Short" Introduction}}
\end{center}
\bigskip
\Large
\textbf{\TeX} (pronouned tech), created by Donald E. Knuth, is a program that is
designed to typeset text and mathematical formulae. \\\\
\textbf{\LaTeX} (pronounced luh-tech) is a kind of dialect of TeX, and is aimed to help beginners/authors to
format their work in a professional manner. \\\\
Together, the writer plays the role of an author, \LaTeX\ plays the role of a book designer, and TeX is the
typesetter. It is quite different from WYSIWYG (what you see is what you get)
word processors like \textbf{Microsoft Word}. You generally cannot see, in real
time, what your document will look like as you work on it in \LaTeX\ since you
have to compile it. \\\\
There are numerous ways to do so - programs like pdflatex and MiKTeX will
compile your \LaTeX\ file into a PDF, and there are packages/extensions in Emacs, Atom, etc. that compile it from within the environment.

\newpage
\begin{center}
	\bf{\Huge{"Short" Introduction (cont.)}}
\end{center}
\bigskip
\Large
In addition, LaTeX is considered to be a programming language. \\
Various aspects of LaTeX are very programm-y. \\\\
For example, you probably already noticed
how \verb|\users| have a section at the top of their base file. \\\\
The \verb|"\usepackage"| command essentially functions as \textbf{import} does
in Python and Java. Users can also write their own commands/syntax.

%Jason's Part (but it's Kevin doing it tho)

%Jason - Advantages and Disadvantages
\newpage
\begin{center}
    \bf{\Huge{Advantages and Disadvantages}}
\end{center}
\bigskip

People who are used to WYSIWYG word processors may question the advantages \LaTeX, but here are a few...

\begin{itemize}
  \item Professionally designed layouts are available
  \item Mathematical formulas are simple and intuitive to insert
  \item Few commands will allow the user to specify the layout of the document.
  \item Additional packages are available in order to enhance or improve upon basic \LaTeX.
  \item Everything ends up appearing very structured and organized.
  \item Free!
\end{itemize}

However, there are still some disadvantages.

\begin{itemize}
    \item Because compilation is necessary, those who are not used to the syntax/structure of \LaTeX may find themselves compiling over and over again.
    \item Not really a great tool if all you want to do is get your ideas on paper quickly
    \item Not really used in fields outside of math, computer science, and physics.
\end{itemize}

%Basics
\newpage
\begin{center}
	\bf{\Huge{The Basics}}
\end{center}
\bigskip
\LARGE
At its core, LaTeX is still a word processor, so if you just type text into a LaTeX file and compile it, something will appear. \\\\
\Large
But there are certain ways you always begin a new document. \\
\begin{verbatim}
	\documentclass[a4paper,12pt]{article}
	\begin{document}
	.....
	\end{document}
\end{verbatim}
\bigskip
Any phrase starting with a backslash LaTeX will try to read as a command instead of plain text. \\\\
Here, the command \verb|\documentclass| allows you to set what kind of document you are creating. \textit{article} is for standard writing,
but there are also options for long reports, books, presentations, letters, etc. \\\\
The main arguments of any command are always enclosed in curly braces \verb|{}|, and the additional arguments are usually inside square brackets \verb|[]|. \\\\

%Environments
\newpage
\begin{center}
	\bf{\Huge{The Basics - Environments}}
\end{center}
\bigskip
\LARGE
The \verb|\begin| and \verb|\end| statements you see indicate the start and end of an environment. \\\\
\Large
Environments in LaTeX somehow affect everything enclosed within the begin and end statements. The simplest example would be something like
the \textit{center} environment. 
\begin{center}
	{\bf It does this to the enclosed text.}
\end{center}
\begin{verbatim}
	\begin{center}
		{\bf It does this to the enclosed text.}
	\end{center}
\end{verbatim}
\bigskip
It's important to note that before and after an environment, a new line is automatically created. \\\\
White space works very differently in LaTeX. Pressing \textbf{Enter} to create a new line in the code doesn't create a new line after you compile. In order to go to the next line, you have to use a special command: either \verb|\newline|, \verb|\break|, or simply \verb|\\|.

%Math
\newpage
\begin{center}
	\bf{\Huge{The Basics - Math}}
\end{center}
\bigskip
\LARGE
LaTeX is used frequently for scientific/mathematic/research papers. It kind of follows that making math pretty is one of LaTeX's greatest strengths. \\\\
\Large
Math equations can be inserted in LaTeX in a few ways. \\\\
{\bf In a paragraph:} \\
We can use the Pythagorean Theorem, $a^2 + b^2 = c^2$, to solve for the hypotenuse. \\
\verb|$a^2 + b^2 = c^2$| \\\\
{\bf In an environment:} \\
Let's take a look at the equation for electric field:
\begin{equation}
	E = \int \frac{kdq}{r^2}
\end{equation}
\begin{verbatim}
	Let's take a look at the equation for electric field:
	\begin{equation}
		E = \int \frac{kdq}{r^2}
	\end{equation}
\end{verbatim}
\bigskip
{\bf Using this special syntax:} \\
Here is the definition of a Riemann sum:
\[S = \sum_{i=1}^{n} f(x_i)\Delta x\]
\begin{verbatim}
	Here is the definition of a Riemann sum:
	\[ S = \sum_{i=1}^{n} f(x_i) \Delta x \]
\end{verbatim}

%Lists
\newpage
\begin{center}
	\bf{\Huge{The Basics - Lists}}
\end{center}
\bigskip
\LARGE
Lists are also very useful and easy to make in LaTeX. \\\\
\Large
They are inserted using different environments. \\\\
{\bf The Unordered List}
\begin{itemize}
	\item Do you like pancakes?
	\item Do you like waffles?
\end{itemize}
\begin{verbatim}
	\begin{itemize}
		\item Do you like pancakes?
		\item Do you like waffles?
	\end{itemize}
\end{verbatim}
\bigskip
{\bf The Ordered List}
\begin{enumerate}
	\item Finish project
	\item ???
	\item Profit
\end{enumerate}
\begin{verbatim}
	\begin{enumerate}
		\item Finish project
		\item ???
		\item Profit
	\end{enumerate}
\end{verbatim}
\vspace{\stretch{1}}
There's more things you can do with lists, but that's enough for now. 

%Tables
\newpage
\begin{center}
	\bf{\Huge{The Basics - Tables}}
\end{center}
\bigskip
\LARGE
It is, of course, also very useful to have access to tables. We are doing research, after all. 
\begin{center}
	\begin{tabular}{|c|c|c|}
		\hline
 		Boys & Girls & Total \\ \hline
 		5 & 17 & 22 \\ \hline 
 		15 & 11 & 26 \\ \hline
	\end{tabular}
\end{center}
\begin{verbatim}
	\begin{center}
		\begin{tabular}{|c|c|c|}
			\hline
 			Boys & Girls & Total \\ \hline
 			5 & 17 & 22 \\ \hline 
 			15 & 11 & 26 \\ \hline
		\end{tabular}
	\end{center}
\end{verbatim}	
\bigskip
\Large
Notice that here we have a nested environment (tabular within a center). This is perfectly legal and everything, I just thought y'all should know that it is. \\\\
LaTeX being LaTeX, everything in the table must be specified. Where and where not to draw a vertical line, a horizontal line, etc. \\\\
While this makes LaTeX tedious at times, it gives you a level of customization far greater than any standard word processor. 

%Kevin's Part

%Intro/Summary 
\newpage
\begin{center}
	\bf{\Huge{The Cool Stuff}}
\end{center}
\bigskip
\LARGE
Now that we've gone over the basics, let's take a look at some of the other cool things you can do with LaTeX!
\begin{itemize}
	\item Graphics
	\item Packages
\end{itemize}
\bigskip
\Large
LaTeX has the capability to do almost anything that you'd need from a word processor. \\\\
In order to implement most of the cool features available in \\ LaTeX, we'll need to use packages. \\\\
To use a package, similar to imports in Java or Python, we simply declare \verb|\usepackage{packagename}| at the top of our file. \\\\
Afterwards, we can use all the functions that package provides. \\\\
There are a few essential packages that most people will end up using. \\\\
For example, the package \textit{fullpage} will set the margins of the paper to normal 1x1 instead of math paper margins.

%Graphicx 
\newpage
\begin{center}
  	\bf{\Huge{Pictures}}
\end{center}
\bigskip
\Large
The easiest way to insert a picture in LaTeX is with the package \textit{graphicx}. \\
\begin{figure}[h]
	\centering
		\includegraphics[scale = 0.5]{heavy_breathing}
\end{figure}
\large
\begin{verbatim}
	\usepackage{graphicx}
	.....
	\begin{figure}[h]
		\centering
		\includegraphics[scale = 0.5]{heavy_breathing}
	\end{figure}
\end{verbatim}
\bigskip
Here, the \verb|[h]| is an additional argument that can be given to the environment. \\\\
h means to display the image \textit{here} on the page. Images enclosed in figure environments will float to the top of the page by default. \\\\
You can also use the arguments $t$ for top, $b$ for bottom, etc. \\\\
And you can use more than one argument in order of preference in case one of them fails: \verb|[htb]|

%SI Unitx 
\newpage
\begin{center}
	\bf{\Huge{SI Units}}
\end{center}
\bigskip
\[x = 1.048\ \si{m} - 1.04\ \si{m} = 0.008\ \si{m}\]
\[F = mg = 0.560\ \si{kg}*9.81\ \si{m/s^2} = 5.49\ \si{N}\]
\[k_{eff} = avg(\frac{F}{x}) = \frac{1}{4}(\frac{5.49\ \si{N}}{0.008\ \si{m}} + \frac{7.94\ \si{N}}{0.017\ \si{m}} + \frac{10.40\ \si{N}}{0.024\ \si{m}} + \frac{8.63\ \si{N}}{0.020\ \si{m}}) = 504.5\ \si{N/m}\]
\bigskip
\begin{verbatim}
	\usepackage{siunitx}
	.....

	\begin{equation}

	x = 1.048\ \si{m} - 1.04\ \si{m} = 0.008\ \si{m} \\

	F = mg = 0.560\ \si{kg}*9.81\ \si{m/s^2} = 5.49\ \si{N} \\

	k_{eff} = avg(\frac{F}{x}) = 
	\frac{1}{4}(\frac{5.49\ \si{N}}{0.008\ \si{m}} +
	\frac{7.94\ \si{N}}{0.017\ \si{m}} +
	\frac{10.40\ \si{N}}{0.024\ \si{m}} +
	\frac{8.63\ \si{N}}{0.020\ \si{m}}) 
	= 504.5\ \si{N/m}

	\end{equation}
\end{verbatim}
\bigskip
Putting a \textbackslash\ after any word or number adds a space. There are different size spaces you can add, such as
\verb|\,| (thin space), \verb|\enspace| (0.5 cm), \verb|\quad| (1 cm), etc.

%Code Snippets 
\newpage
\begin{center}
	\bf{\Huge{Code Snippets}}
\end{center}
\bigskip
\Large
Here's a code snippet which contains the code needed to insert a code snippet into LaTeX.
\begin{lstlisting}
	\usepackage{listings}
	\usepackage{xcolor}
	.....
	\lstset{
		language=[LaTeX]TeX,
		breaklines=true,
		basicstyle=\tt\scriptsize,
		keywordstyle=\color{blue},
		identifierstyle=\color{magenta}
	}
	.....
	\begin{lstlisting}
		.....
\end{lstlisting}
The \textit{listings} package provides us with the new lstlisting environment, and a way to edit it: \textit{lstset}. \\\\
\textit{xcolor} gives us access to the values "blue" and "magenta".

%Johnny again - Applications
\newpage
\begin{center}
	\bf{\Huge{Applications}}
\end{center}
\bigskip

\begin{itemize}
  \item School papers
  \item Notes
  \item Scholarly Magazine/Journal
  \item Competitions (for Math, Science)
\end{itemize}
\end{document}
