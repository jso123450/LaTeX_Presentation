\documentclass[a4paper,12pt]{article}
\usepackage{fullpage}
\usepackage{listings}
\usepackage{xcolor}

%This allows us to insert LaTeX code snippets
\lstset{
    language=[LaTeX]TeX,
    breaklines=true,
    basicstyle=\tt\scriptsize,
    keywordstyle=\color{blue},
    identifierstyle=\color{magenta},
}

\title{LaTeX}
\author{Jason Shin \and Johnny So \and Kevin Yan}

\begin{document}
\maketitle

%Johnny's Part
\newpage
\begin{center}
  \bf{\Huge{Short Introduction}}
\end{center}
\bigskip
\Large
\textbf{\TeX} (pronouned tech), created by Donald E. Knuth, is a program that is
designed to typeset text and mathematical formulae. \textbf{\LaTeX} (pronounced
luh-tech) is a kind of dialect of TeX, and is aimed to help beginners/authors to
format their work in a professional manner. Together, the writer plays the role
of an author, \LaTeX plays the role of a book designer, and TeX is the
typesetter. It is quite different from WYSIWYG (what you see is what you get)
word processors like \textbf{Microsoft Word}. You generally cannot see, in real
time, what your document will look like as you work on it in \LaTeX since you
have to compile it. There are numerous ways to do so - stuff like pdflatex will
compile your \LaTeX file into a PDF, and there are packages/extensions in Emacs
that compile it from within the environment.

%Kevin's Part

%Intro/Summary Page
\newpage
\begin{center}
	\bf{\Huge{The Cool Stuff}}
\end{center}
\bigskip
\Large
Now that we've gone over the basics, let's take a look at some of the other cool things you can do with LaTeX!
\begin{itemize}
	\item Tables and graphs
	\item Pictures
	\item Commands
	\item Environments
	\item Packages
\end{itemize}
\medskip
\large
What is a research paper without any sort of graphics, after all? \\\\
However, in order to use most of the cool features in LaTeX, we'll need to use packages. \\\\
TO use a package, similar to imports in Java or Python, we simply declare \usepackage{packagename} at the top of our file. \\\\
Afterwards, we can use all the functions that package provides. \\\\
There are a few essential packages that most people will end up using. \\\\
For example, textit{fullpage} will set the margins of the paper to normal 1x1 instead of math paper margins.

%Graphicsx Page
\newpage
\begin{center}
  \bf{\Huge{Pictures}}
\end{center}
\bigskip
\Large
The easiest way to insert a picture into LaTeX is with the package \textit{graphicsx}. \\\\
\large

\end{document}
